
\chapter*{O porquê desse texto}
\label{chap:intro}

\lettrine[lines=3]{\color{azulUFRB} \initial T}{alvez essa seja} a quarta (ou 
quinta, realmente não lembro) tentativa de escrever um texto sobre \hologo{LaTeX} 
que seja sucinto, introdutório e que aponte um caminho seguro para quem deseja 
se aprofundar em referências consolidadas sobre o tema.
Não tenho a pretensão de seja um \textsf{manual} ou \textsf{tutorial} sobre 
\hologo{LaTeX}. 
Ele é apenas um \textsf{texto de apoio} a um microcurso (denomino ``micro''; 
pois, geralmente, é menos do que \qty{4}{h} de curso --- um completo absurdo, 
diga-se de passagem \emoji{smile}).
A leitura de um manual/tutorial consolidado é \textsf{fundamental} se você 
deseja usar esse sistema de composição tipográfica em seus textos (farei algumas 
indicações mais adiante). 

\begin{center}
  \begin{PostItNote}[Render=tikz, Color=douradoUFRB]
    \sffamily
    Embora você possa ler esse texto e fruir de uma noção inicial do uso do 
    \textrm{\hologo{LaTeX}}, o ideal seria a participação no microcurso. 
  \end{PostItNote}
\end{center}

Do exposto, vamos delinear alguns procedimentos para que a leitura desse texto 
seja a mais proveitosa possível. 

É meu desejo que, ao final do microcurso (ou desta leitura) você esteja apto a 
reproduzir a seguinte estrutura de um artigo:

\begin{center}
  \linkExt{%
    https://icaro-freire.github.io/projeto_artigo/main.pdf
  }{Artigo Genérico Isento de Sentido}
\end{center}

Obviamente, o conteúdo do artigo (cômico em certas partes) em questão não está 
estruturado de forma linear. 
Essa não foi a intenção. 
O desejo é que você saiba reproduzir a \textsf{estrutura} de um artigo genérico 
em Matemática: referências cruzadas; estruturas de teoremas, proposições, 
definições, etc.; inserção de figuras; estruturas matemáticas básicas como 
matrizes, potências, integrais, derivadas, etc.; alinhamento de equações; 
nomeação de equações; lista; tabelas e referências bibliográficas.

Para tanto, a organizaçao desse texto é a seguinte: darei uma visão geral das 
estrururas e comandos básicos usados no artigo; e, depois, o ``passo a passo''
da construção do artigo citado. 

Mais especificamente:

\begin{description}
  \item[\textbs{Capítulo~\ref{cap:latex}}] 
       Basicamente, desejo abordar as nucances do que seja o \hologo{LaTeX}; 
       sua filosofia; estruturação da classe \textit{article}; e, como podemos 
       organizar os mais diferentes arquivos numa produção de um artigo nesse 
       sistema. 
  \item[\textbs{Capítulo~\ref{cap:modo-texto}}] 
       A intenção neste capítulo é expor alguns dos comandos ou ambientes 
       frequentemente usados no \textsf{modo texto}: tamanho e estilo de fontes;
       uso de tabelas e figuras; configurações das legendas; referências cruzadas; 
       referências no padrão da \textsc{abnt}; como criar novos comandos; etc. 
  \item[\textbs{Capítulo~\ref{cap:modo-math}}] 
       Neste penúltimo capítulo, pretendo abordar estruturas e ambientes para 
       escrita matemática. 
       Obviamente é um tópico bastante amplo e que apenas será tangenciado;
       na esperança de que os leitores percebam a beleza que esse sistema pode 
       oferecer na construção de textos que envolvam espressões matemáticas 
       (embora tal beleza não seja exclusiva ao modo matemático).
  \item[\textbs{Capítulo~\ref{cap:passo-a-passo}}] 
       Por fim, mostro o passo a passo de como construir o \textsf{Artigo Genérico}
       citado mais acima. 
\end{description}

É importante ressaltar que, para esse microcurso, usaremos uma a plataforma 
\textit{online} \linkExt{https://www.overleaf.com/}{Overleaf} para escrevermos o 
\textsf{Artigo Genérico}. 
Tal plataforma disponibiliza muitas ferramentas que, no meu entendimento, pode 
ser icentivadoras num primeiro contato com o \hologo{LaTeX} (não precisamos 
lidar com erros de instalação; a renderização do texto pode ser quase automática; 
diversos \textit{templates} que podem ser acessados com muita dacilidade; etc.). 
Para saber mais sobre essa plataforma, inclusive como configurá-la para nossos 
objetivos, acesse o Apêndice~\ref{apend:overleaf}.

Espero que esse esforço seja útil para aqueles que buscam conhecer esse 
sistema de composição tipográfica tão surpreendente. 

\begin{flushright}
  Amargosa, \today. 

  \caligrafica \LARGE
  Ícaro Vidal Freire
\end{flushright}




